% Paket um vordefinierte Texte (z.B. "Inhaltsverzeichnis") auf Deutsch zu übersetzen
\usepackage[ngerman]{babel}

% Paket um Schriftarten festzulegen (für XeLaTeX)
\usepackage{fontspec}

% serifenfreie Schriftart Arial festlegen
%\setsansfont{Arial}

% serifenfreie Schriftvariante verwenden
\renewcommand{\familydefault}{\sfdefault}

% Paket um Grafiken (JPG, PNG, PDF) einzubinden
\usepackage{graphicx}

% Paket für Zeilenabstand
\usepackage{setspace}

% Paket für korrekte Anführungszeichen
\usepackage{csquotes}

% Paket für selbst definierte Kopf- und Fusszeilen
\usepackage{scrlayer-scrpage}

% Paket für Zitate und Bibliografie
%\usepackage{biblatex}

%\addbibresource{refs.bib}

% Paket zum Erzeugen von Platzhaltertext
\usepackage{lipsum}

% Used for including code snippets
\usepackage{listings}
\usepackage{xcolor}
\newcommand*{\listingFont}{\ttfamily}
\lstdefinelanguage{QHS}
{
    %morekeywords={\textgreater\textgreater},
    %morecomment=[l]{//}, % l is for line comment
    morecomment=[s]{/*}{*/}, % s is for start and end delimiter
    morestring=[b]", % defines that strings are enclosed in double quotes (LiteralCode)
    %alsoletter=\textgreater
}

\definecolor{codegreen}{rgb}{0,0.6,0}
\definecolor{codegray}{rgb}{0.5,0.5,0.5}
\definecolor{codebrown}{HTML}{AB7763}
\definecolor{codeviolet}{HTML}{785175}
\definecolor{backcolour}{HTML}{CCD6E8}
\lstdefinestyle{QHSstyle}
{
    %backgroundcolor=\color{backcolour},   
    commentstyle=\color{codegreen},
    keywordstyle=\color{codeviolet},
    numberstyle=\tiny\color{codegray},
    stringstyle=\color{codebrown},
    basicstyle=\listingFont\footnotesize,
    frame=single,
    frameround=tttt,
    escapechar=\%,
    %numbers=left,
    %stepnumber=1,
}

\lstset{style=QHSstyle}

% Used for better float placement
\usepackage{float}

% Used for better tabulators
\usepackage{tabularx}

% Used for data figures
\usepackage{pgfplots}
% Externalize pgfplots for faster compilation
%\usepgfplotslibrary{external}
%\tikzexternalize


