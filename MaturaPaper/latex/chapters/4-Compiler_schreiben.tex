\chapter{Schreiben der Compiler}

\section{THS Compiler}
Der THS Compiler folgt dem theoretischen Aufbau eines Compilers und besteht aus Lexer, Parser und Code Generator. Als Parser wird ein Predictive Descent Parser verwendet.
Der Code Generator arbeitet auf dem Abstract Syntax Tree mithilfe eines Visitor Patterns. Die Semantic Analysis wird während der Code Generation durchgeführt. Geschrieben ist der Compiler in C++ und liefert x86 Assembly nach NASM Syntax.

\section{QHS Compiler}
Der QHS Compiler beginnt wie ein theoretischer Compiler mit Lexical Analysis. Die Syntax Analysis wird jedoch übersprungen. Genauso wie der THS Compiler ist auch der QHS Compiler in C++ geschrieben und
generiert x86 Assembly nach NASM Syntax.

\subsection{Lexical Analysis}
Der QHS Compiler Lexer Unterscheidet nur zwischen 4 Tokens: Whitespaces, Identifiers, Instructions und LiteralCode. \textbf{Die} Lexical Grammar von THS definiert die Tokens wie folgt:

\begin{table}[h]
    \centering
    \begin{tabular}{l|l}
    \textless{}whitespace\textgreater{}    & SPACE | NEWLINE | TAB                                           \\ \hline
    \textless{}identiferChar\textgreater{} & {[}\textasciicircum \# " \textless{}whitespace\textgreater{}{]} \\ \hline
    \textless{}identifier\textgreater{}    & \textless{}identiferChar\textgreater{}*                         \\ \hline
    \textless{}instruction\textgreater{}   & \# \textless{}identiferChar\textgreater{}*                      \\ \hline
    \textless{}literalCode\textgreater{}   & ".*"                                                         
    \end{tabular}
\end{table}

\subsection{Code Generation}