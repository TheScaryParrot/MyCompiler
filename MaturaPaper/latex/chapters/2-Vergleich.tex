\chapter{Vergleich der Compiler} \label{cha:2-Vergleich}
Um die Leistung meines Compilers zu testen, werde ich ihn mit zwei weiteren Compilern vergleichen.

\begin{enumerate}
\item
Der \textit{QHScompiler} ist der von mir nach meiner Idee entwickelte Compiler. Seine genaue Funktionsweise wird in Abschnitt \ref{cha:4-QHS_Compiler} ausgeführt.
Er ist in C++ geschrieben und generiert x86 Assembly mit NASM Syntax aus meiner Programmiersprache QHS.

\item
\textit{GCC} ist ein gebräuchlichste Compiler für die Programmiersprache C. Veröffentlicht im Jahre 1987 wird GCC bis heute weiterentwickelt und ermöglicht inzwischen auch die Kompilierung von C++, Rust, Fortran, usw.
[...] <- Add if data is actually useful

\item
Der \textit{THScompiler} repräsentiert in diesem Vergleich einen traditionellen von mir geschrieben Compiler. Im Gegensatz zu GCC kann der THScompiler weniger.
Er dient daher als realistische Konkurrenz zum QHScompiler und wird verwendet, um zu testen, ob meine Idee taugt. Der THScompiler folgt dem in Abschnitt \ref{cha:3-Tradional_Compiler} beschriebenen traditionellen Aufbau eines Compilers.
Geschrieben ist der THScompiler in C++ und liefert x86 Assembly mit NASM Syntax.
\end{enumerate}

\section{Anforderungen an die Compiler}
Für die von mir entwickelten Compiler habe ich folgende Mindestanforderungen gestellt:

\begin{table}[H]
    \centering
    \caption{Anforderungen an die Compiler}
    \vspace{3mm} % Adjust the height of the space between caption and tabular
    
    \begin{tabular}{l|l}
    Output als Assembly Code        & Die Output-Sprache muss Assembly Code sein                               \\
    C-like Syntax                   & Die Input-Sprache muss einen C-ähnlichen Syntax aufweisen                \\
    Variablen und Funktionen        & Lokale und globale Variablen sowie Funktionen müssen unterstützt werden  \\
    Branching und Loops             & If-Statements sowie While-Loops müssen umsetzbar sein
    \end{tabular}
\end{table}


\section{Kriterien des Compiler-Vergleichs}
Zum Schluss werden die drei Compiler nach folgenden Kriterien bewertet und verglichen.

\begin{table}[H]
    \centering
    \caption{Vergleichskriterien der Compiler}
    \vspace{3mm} % Adjust the height of the space between caption and tabular

    \begin{tabular}{l|l}
    Geschwindigkeit des Output-Codes    & Wie lange dauert die Ausführeung des Output-Codes?                \\
    Geschwindigkeit der Kompilierung    & Wie lange dauert die Kompilierung von Code?                       \\
    Benutzerfreundlichkeit              & Wie einfach ist die Verwendung des Compilers?                     \\
    Offenheit für Erweiterung           & Wie einfach kann die Input-Sprache erweitert werden?                                 
    \end{tabular}
\end{table}

