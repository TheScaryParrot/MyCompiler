\chapter{Vergleich der Compiler} \label{cha:2-Vergleich}
Um die Leistung meines alternativen Aufbaus eines Compilers zu testen, werde folgende drei Compiler vergleichen:

\begin{enumerate}
\item
Der \textit{QHScompiler} ist der von mir nach meinem Aufbau entwickelte Compiler. Seine genaue Funktionsweise wird in Abschnitt \ref{cha:4-QHS_Compiler} ausgeführt.
Er ist in C++ geschrieben und verwendet meine eigene Programmiersprache QHS als Eingabesprache. 

\item
\textit{GCC} ist der gebräuchlichste Compiler für die Programmiersprache C. Veröffentlicht im Jahre 1987 wird GCC bis heute weiterentwickelt und ermöglicht inzwischen auch die Kompilierung von C++, Rust, Go, usw.
Passend dazu lautet das Akronym GCC ausgeschrieben: GNU Compiler Collection.
Für diesen Vergleich repräsentiert GCC den traditionellen Compileraufbau.

\item
Der \textit{THScompiler} ist ebenfalls ein traditioneller Compiler. Der Unterschied zu GCC liegt jedoch darin, dass der THScompiler von mir selbst entwickelt wurde. 
Er ist dadurch deutlich weniger optimiert und umfasst weniger Funktionen. In der Komplexität entspricht der THScompiler ungefähr dem QHScompiler.
Der THScompiler dient mit ähnlichem Arbeitsaufwand, Optimierung und Niveau der Programmierung als "realistische" Konkurrenz zum QHScompiler.
Geschrieben ist der THScompiler in C++ und kompiliert aus meiner eigenen Programmiersprache THS.
\end{enumerate}

Für die von mir entwickelten Compiler habe ich folgende Mindestanforderungen gestellt:

\begin{table}[H]
    \centering
    \caption{Anforderungen an die Compiler}
    \label{tab:requirements}
    \vspace{3mm} % Adjust the height of the space between caption and tabular
    
    \begin{tabular}{l|l}
    Ausgabe als Assembly Code       & Die Ausgabesprache muss Assembly Code sein                               \\
    C ähnlicher Syntax              & Die Eingabesprache muss einen C ähnlichen Syntax aufweisen               \\
    Variablen und Funktionen        & Lokale und globale Variablen sowie Funktionen müssen unterstützt werden  \\
    Branching und Loops             & If-Statements und einfache Loops müssen umsetzbar sein
    \end{tabular}
\end{table}

Die Kriterien des Vergleichs sind in Tabelle \ref{tab:criteria} aufgelistet.
%Zum Schluss werden die drei Compiler nach folgenden Kriterien bewertet und verglichen.

\begin{table}[H]
    \centering
    \caption{Vergleichskriterien der Compiler}
    \label{tab:criteria}
    \vspace{3mm} % Adjust the height of the space between caption and tabular

    \begin{tabular}{l|l}
    Geschwindigkeit der Kompilierung    & Wie lange dauert die Kompilierung von Code?                       \\
    Geschwindigkeit der Ausgabedatei    & Wie lange dauert die Ausführung der Ausgabedatei?                 \\
    Umgang mit fehlerhaftem Code        & Wie geht der Compiler mit fehlerhaften Eingabedateien um?         \\
    Offenheit für Erweiterung           & Wie einfach kann die Eingabesprache erweitert werden?                                 
    \end{tabular}
\end{table}

