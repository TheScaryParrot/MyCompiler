\chapter{Vergleich der Compiler} \label{cha:2-Vergleich}
Um meinen alternativen Compileraufbau zu testen, werde ich folgende drei Compiler vergleichen:

\begin{enumerate}
\item
Der QHScompiler ist der von mir nach meinem Aufbau entwickelte Compiler. Seine genaue Funktionsweise wird in Kapitel \ref{cha:4-QHS_Compiler} ausgeführt.
Er ist in C++ geschrieben und verwendet meine eigene Programmiersprache QHS als Eingabesprache. 

\item
Bei der Veröffentlichung im Jahr 1987 stand das Akronym GCC noch für GNU C Compiler. Heute ist GCC die GNU Compiler Collection, eine Sammlung an Compilern für verschiedene Programmiersprachen, darunter C, C++, Rust, etc.
Für diesen Vergleich verwende ich den C Compiler aus der GNU Compiler Collection (Version 12.4).
Jede weitere Verwendung der Abkürzung GCC referiert auf diese Version des C Compilers.
Für diesen Vergleich repräsentiert GCC den traditionellen Compileraufbau.

%GCC ist ein Compiler für die Programmiersprache C. Veröffentlicht im Jahre 1987 wird GCC bis heute weiterentwickelt und ermöglicht inzwischen auch die Kompilierung von C++, Rust, Go, usw.
%Passend dazu lautet das Akronym GCC ausgeschrieben: GNU Compiler Collection.
%Für diesen Vergleich repräsentiert GCC den traditionellen Compileraufbau.

\item
Der THScompiler ist ebenfalls ein traditioneller Compiler. Der Unterschied zu GCC liegt jedoch darin, dass der THScompiler von mir selbst entwickelt wurde. 
Er ist dadurch deutlich weniger optimiert und umfasst weniger Funktionen. In der Komplexität entspricht der THScompiler ungefähr dem QHScompiler.
Der THScompiler dient mit ähnlichem Arbeitsaufwand, Optimierung und Niveau der Programmierung als "realistische"{} Konkurrenz zum QHScompiler.
Geschrieben ist der THScompiler in C++ und kompiliert aus meiner eigenen Programmiersprache THS.
\end{enumerate}

An die von mir entwickelten Compiler habe ich folgende Mindestanforderungen gestellt:

\begin{table}[H]
    \centering
    \caption{Anforderungen an die Compiler}
    \label{tab:requirements}
    \vspace{3mm} % Adjust the height of the space between caption and tabular
    
    \begin{tabular}{l|l}
    Ausgabe als Assembly-Code       & Die Ausgabesprache muss x86 Assembly sein                                \\
    C ähnlicher Syntax              & Die Eingabesprache muss einen C ähnlichen Syntax aufweisen               \\
    Variablen und Funktionen        & Lokale und globale Variablen sowie Funktionen müssen unterstützt werden  \\
    Verzweigungen und Schleifen     & Verzweigungen und Schleifen müssen umsetzbar sein
    \end{tabular}
\end{table}

%Die Kriterien des Vergleichs sind in Tabelle \ref{tab:criteria} aufgelistet.
%Zum Schluss werden die drei Compiler nach folgenden Kriterien bewertet und verglichen.
Dabei sollen die Compiler in folgenden Kriterien verglichen werden:

\begin{table}[H]
    \centering
    \caption{Vergleichskriterien der Compiler}
    \label{tab:criteria}
    \vspace{3mm} % Adjust the height of the space between caption and tabular

    \begin{tabular}{l|l}
    Geschwindigkeit der Kompilierung    & Wie lange dauert die Kompilierung?                                \\
    Geschwindigkeit der Ausgabedatei    & Wie lange dauert die Ausführung der Ausgabedatei?                 \\
    Umgang mit fehlerhaftem Code        & Wie geht der Compiler mit Fehlern in der Eingabedatei um?         \\
    Offenheit für Erweiterung           & Wie einfach kann die Eingabesprache erweitert werden?                                 
    \end{tabular}
\end{table}

