\chapter{Fazit}
Der QHScompiler hat im Vergleich nicht besonders gut abgeschnitten. Er ist sowohl in der Geschwindigkeit der Kompilierung als auch bei der eines kompilierten Programmes einem traditionellen Compiler unterlegen.
Zudem ist der QHScompiler auch nicht besonders benutzerfreundlich und nicht wirklich angenehm zum Verwenden. Als einziger Vorteil lässt sich seine Möglichkeit zur Erweiterung sehen.
Daher lässt sich klar sagen, zum Compiler ist der QHScompiler nicht besonders gut geeignet. Jedoch ist deswegen der QHScompiler nicht gleich nutzlos. Mit etwas Optimierung könnte er nämlich immer noch als praktisches Tool dienen. 
Und zwar fürs Schreiben von Assembly Code. Man könnte mit LiteralCode weiterhin den grössten Teil des Codes normal in Assembly schreiben, sich häufig wiederholende Code Stücke jedoch mit Identifiern abkürzen.
Besonders für spezialisierte Prozessor Architekuren mit spezifischen Instructionsets könnte der QHScompiler eine einfachere Alternative zu einem kompletten Compiler darstellen.
Somit ist der QHScompiler zwar keine bahnbrechende Idee, die die modernen Compiler in den Schatten stellt, aber zumindest ein kleines nützliches Tool, dass ich vielleicht das ein oder andere Mal noch verwenden werde.
