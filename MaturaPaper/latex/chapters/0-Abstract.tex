\begin{abstract}
Während ich mich mit der Theorie hinter modernen Compilern befasste, stellte ich mir häufig die Frage: Wieso sind Compiler so, wie sie sind?
Ich nahm mir daraufhin vor, selbst einen alternativen Aufbau für Compiler zu entwickeln, mit dem Ziel zu verstehen, wieso sich das traditionelle Compilermodell so gut bewährt.
In dieser Maturaarbeit werde ich zuerst den traditionellen Aufbau von Compilern kurz beschreiben und mein eigenes alternatives Modell vorstellen.
Die Umsetzung meiner Idee werde ich anhand eines selbstgeschriebenen alternativen Compilers erläutern.
Dieser alternative Compiler wird daraufhin mit zwei traditionellen Compilern, wovon ich ebenfalls einen selbst geschrieben habe, verglichen.
Zum Schluss werde ich die Resultate des Vergleichs einordnen und daraus Vor- und Nachteile des traditionellen Compileraufbaus ableiten.
\end{abstract}
