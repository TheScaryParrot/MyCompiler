\chapter{Meine Idee}
Ein Compiler ein äusserts komplexes Programm, mit vielen verschiedenen Schritten. Jedoch ist die zugrundeliegende Aufgabe gar nicht so kompliziert.
Man braucht ja nur, ein Dokument mit Text der bestimmten Regeln folgt, in Text mit anderen Regeln verwandeln. Natürlich ist dies etwas salopp ausgedrückt, trotzdem fragte ich mich,
ob es nicht möglich sei einen viel einfacheren Compiler zu schreiben. Als Grundidee 

\section{Vergleich der Compiler}
Der THS Compiler folgt dem theoretischen Aufbau eines Compilers und besteht aus Lexer, Parser und Code Generator. Als Parser wird ein Predictive Descent Parser verwendet.
Der Code Generator arbeitet auf dem Abstract Syntax Tree mithilfe eines Visitor Patterns. Die Semantic Analysis wird während der Code Generation durchgeführt. Geschrieben ist der Compiler in C++ und liefert x86 Assembly nach NASM Syntax.

Genauso wie der THScompiler ist auch der QHScompiler in C++ geschrieben und generiert x86 Assembly nach NASM Syntax.

\subsection{Anforderungen an die Compiler}
Um einen \textbf{fairen} Vergleich zu ermöglichen, müssen die Compiler folgende Anforderungen erfüllen.

\begin{table}[h]
    \begin{tabular}{l|l}
    Output als Assembly Code     & Die Output-Sprache muss Assembly Code sein                               \\
    C-like Syntax                & Die Input-Sprache muss einen C-ähnlichen Syntax aufweisen                \\
    Variablen und Funktionen     & Lokale und globale Variablen sowie Funktionen müssen unterstützt werden  \\
    Benutzerdefinierte Datatypes & Benutzerdefinierte Datatypes müssen unterstützt werden                                 
    \end{tabular}
\end{table}

Die Anforderungen machen dass Compiler gleich komplex.

\subsection{Kriterien des Vergleichs}
Die Compiler werden nach folgenden Kriterien bewertet und verglichen.

\begin{table}[h!]
    \begin{tabular}{l|l}
    Geschwindikeit des Output-Codes     & Wie schnell wird der Output-Code ausgeführt?                      \\
    Geschwindikeit der Compilation      & Wie lange dauert die Compilation von Code?                        \\
    Benutzerfreundlichkeit              & Wie einfach ist die Verwendung des Compilers?                     \\
    Möglichkeit für Erweiterung         & Wie einfach ist die Input-Sprache zu erweitern?                                 
    \end{tabular}
\end{table}