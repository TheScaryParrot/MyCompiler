\chapter{Auswertung}

Wie bereits in Kapitel \ref{cha:3-Meine_Idee} beschrieben wurde, sollen drei Compiler QHScompiler, THScompiler und GCC sowie deren dazugehörigen Sprachen QHS, THS und C verglichen werden.
Diese werden in Geschwindigkeit der Compilation, Geschwindigkeit eines compilierten Programmes, Benutzerfreundlichkeit und Offenheit für Erweiterung bewertet. 

\section{Geschwindigkeit der Compilation}

\begin{center}
\begin{tikzpicture}
    \begin{axis}[
        enlargelimits=false,
        xlabel=File Size,
        xmode=log,
        log basis x=10,
        ylabel=Compile Time (ms),
        ymode=log,
        log basis y=10,
        tick label style={font=\bfseries\large},grid=major,
        legend style={at={(0.5, 1.2)}, anchor=north,legend columns=-1},
    ]
    \addplot[
        smooth,
        color=blue,
        mark=square,
        mark size=2.9pt]
    table [col sep=comma]
    {resources/data/compilespeed_qhs.csv};
    \addplot[
        smooth,
        color=red,
        mark=square,
        mark size=2.9pt]
    table [col sep=comma]
    {resources/data/compilespeed_ths.csv};
    \addplot[
        smooth,
        color=green,
        mark=square,
        mark size=2.9pt]
    table [col sep=comma]
    {resources/data/compilespeed_c.csv};
    \legend{QHS, THS, C}
    \end{axis}
\end{tikzpicture}
\end{center}

\section{Geschwindigkeit eines Programmes}


\section{Benutzerfreundlichkeit}


\section{Offenheit für Erweiterung}