\chapter{Introduction}
In der Informatik beschreibt \textit{Compiler} ein Programm, das Code aus einer Programmiersprache in eine andere übersetzt. In dieser Hinsicht gleichen Compiler Übersetzern für Menschensprache.
Genauso wie ein Übersetzer für die Kommunikation zwischen zwei verschiedensprachigen Menschen nötig ist, braucht man Compiler um die Kommunikation zwischen Mensch und Computer zu ermöglichen oder zumindest zu vereinfachen.
Grundsätzlich ist es mithilfe einer Assembly Sprache möglich ohne Compiler einem Computer Befehle zu geben, jedoch ist dies aufwendig und nicht gerade simpel. 
Compiler ermöglichen das Übersetzten von verständlicheren Programmiersprachen zu Assembly und sind daher für die Informatik essentiel.
Compiler unterscheiden sich jedoch grundsätzlich von Übersetzern in der Erwartungshaltung, die an sie gestellt wird. Menschensprache ist sehr komplex und nicht immer besonders eindeutig. 
Programmiersprachen hingegen sind so definiert, dass sie möglichst keinen Raum für Missverständnisse oder Ungenauigkeit lassen. Genauso muss auch ein Compiler exakt und fehlerfrei übersetzten.
Neben fehlerfrei muss die Kompilierung auch möglichst schnell sein. Dasselbe gilt natürlich auch für den resultierenden Output-Code. Dieser sollte möglichst optimal generiert werden, um die schlussendliche
Ausführungsdauer so kurz wie möglich zu halten. Und falls sich doch einmal ein Fehler im Input-Code befindet, sollten diese verständlich gemeldet werden. Compiler sind also durchaus keine simplen Programme und daher auch bis heute noch
ein aktives Forschungsgebiet. In folgendem Text werde ich die Idee, Entwicklung und schlussendliche Auswertung eines von mir erdachten alternativen Ansatzes für den Aufbau eines Compilers beschreiben. 
